\section{Statements}
The generalization of statements over container type uses different quantification
of the predicates and elements in its meaning.

Where unary parametricity is the forall predicate forall elements statement
which states that every predicate of the type arguments is satisfied for 
all elements in the container, other predicates using a notation
of exists are imaginable.

Especially the exists predicate exists element statement will be useful.
This statement postulates that for one of the predicate there is 
an element in the container satisfying the predicate.
For example $\exists\exists_{\text{List}}~X~P_X~xs$ states for a list $xs:\text{List}~X$
that there is an element $x\in xs$ with $P_X~x$.

Additionally, other combinations such as a $\exists\forall$ or $\forall\exists$ statement
are possible.

\subsection{Example statements}

We present the statement for common container type together with
their intuitive meaning.

\subsubsection{Product}
% X*X
% X*Y
We first look at a simpler type of products with only one type
argument.

\[\text{Prod1}~X~::=~C~x_1~x_2\]

As there is only one type argument $\forall Q_2$ and $\exists Q_2$ are the same statement.

\begin{center}
\begin{infrule}
P_X~x_1
P_X~x_2
===
Q\forall_{\text{Prod1}}~X~P_X~(C~x_1~x_2)
\end{infrule}
\end{center}

\begin{infrule}
P_X~x_1
===
Q\exists_{\text{Prod1}}~X~P_X~(C~x_1~x_2)
\end{infrule}
\begin{infrule}
P_X~x_2
===
Q\exists_{\text{Prod1}}~X~P_X~(C~x_1~x_2)
\end{infrule}

A more interesting type is the well known product type:
\[ \text{Prod}~X~Y~::=~\text{pair}~x~y \]
We write $(x,y)$ as short notation for $\text{pair}~x~y$

In $\forall\forall$ both arguments $x$ and $y$ has to satisfy 
their corresponding predicates in all cases.
Therefore, every instance $x$ of type $X$ is accompanied by a proof of $P_X~x$ in the 
constructors of $\forall\forall$. The proof for $y:Y$ are added analogously.

\begin{center}
\begin{infrule}
P_X~x
P_Y~y
===
\forall\forall_{\text{Prod}}~X~P_X~Y~P_Y~(x,y)
\end{infrule}
\end{center}

In contrast $\exists\exists$ only requires one of the predicates to be fulfilled by one element.
Therefore, in one constructor the argument $x:X$ is augmented with a proof $P_X~x$.
In another constructor a proof of $P_Y~y$ is required.

\begin{infrule}
P_X~x
===
\exists\exists_{\text{Prod}}~X~P_X~Y~P_Y~(x,y)
\end{infrule}
\begin{infrule}
P_Y~y
===
\exists\exists_{\text{Prod}}~X~P_X~Y~P_Y~(x,y)
\end{infrule}

As there are only one instance of each type in the single constructor of Prod,
the $\forall\exists$ statement is equivalent to $\forall\forall$.
Similarly $\exists\forall$ is the same as $\exists\exists$.

\subsubsection{List}

\[xs: \text{List}~X~::=~[~]~|~x::xs\]

The list container has one constructor with recursion under which 
the statements need to propagate their meaning in order to reason over
all elements in the lists.

As there is only one type $\exists\forall$ coincides with $\forall\forall$ and 
$\forall\exists$ coincides with $\exists\exists$.

The $\forall\forall$ statements ensure that every instance of $x:X$ is accompanied by a proof $P_X~x$.
The recursion in the statement follows the recursion of list to ensure that all elements are covered.

\begin{infrule}
====
\forall\forall_{\text{List}}~X~P_X~[~]
\end{infrule}
\begin{infrule}
P_X~x
\forall\forall_{\text{List}}~X~P_X~xs
====
\forall\forall_{\text{List}}~X~P_X~(x::xs)
\end{infrule}

The $\exists\exists$ can not be fulfilled for the empty list as there is no
instance $x$ to fulfill the only predicate $P_X$.
For the cons constructor the duality of $\forall$ and $\exists$ or rather between
$\land$ and $\lor$ can be observed.
Whereas in $\forall\forall$ two arguments are added requiring that $x$ satisfies $P_X$
and recursively all other elements in the list fulfill $P_X$,
$\exists\exists$ has two constructor requiring only one of the two cases each.
If the new element $x$ added by cons fulfills $P_X$, then there is an element 
fulfilling $P_X$ in the list and the $\exists\exists$ statement holds.
In the other case, we have the assumption that there is an element fulfilling $P_X$ 
in the remaining list $xs$.

\begin{infrule}
P_X~x
====
\exists\exists_{\text{List}}~X~P_X~(x::xs)
\end{infrule}
\begin{infrule}
\exists\exists_{\text{List}}~X~P_X~xs
====
\exists\exists_{\text{List}}~X~P_X~(x::xs)
\end{infrule}


\subsubsection{Complex typ}

% Recursion
% Two types
% Mult constructors with arguments
% multiple arguments of one type

A more complex type can be considered with multiple constructors and type parameters.
The observed behaviour combines the simpler cases seen before.

\[ c:\text{Complex}~X~Y~::=~A~x~|~B~x~y_1~y_2~|~C~y~c\]

For the constructors $A$ and $B$ the $\forall\forall$ handles the arguments
by requiring an proof of the corresponding predicates.
The recursion is handled by recursion of the statement.

\begin{infrule}
P_X~x
====
\forall\forall_{\text{Complex}}~X~P_X~Y~P_Y~(A~x)
\end{infrule}
\begin{infrule}
P_X~x
P_Y~y_1
P_Y~y_2
====
\forall\forall_{\text{Complex}}~X~P_X~Y~P_Y~(B~x~y_1~y_2)
\end{infrule}
\begin{center}
\begin{infrule}
P_Y~y
\forall\forall_{\text{Complex}}~X~P_X~Y~P_Y~c
====
\forall\forall_{\text{Complex}}~X~P_X~Y~P_Y~(C~y~c)
\end{infrule}
\end{center}


An instance constructed by $A$ can not satisfy the $\forall\exists$ statement
as $P_Y$ is not satisfied by an element.
For the $B$ constructor $x$ has to satisfy $P_X$ but for $P_Y$ one of the two 
given elements of type $Y$ can be chosen.
An object constructed using $C$ fulfills the $\forall\exists$ statement if 
the recursive instance $c$ has elements fulfilling $P_X$ and $P_Y$ expressed by
recursion. As there is no argument of type $X$, $P_X$ has to be satisfied by an element in $c$.
Therefore, the second case for $C$ is superfluous.

\begin{infrule}
P_X~x
P_Y~y_1
====
\forall\exists_{\text{Complex}}~X~P_X~Y~P_Y~(B~x~y_1~y_2)
\end{infrule}
\begin{infrule}
P_X~x
P_Y~y_2
====
\forall\exists_{\text{Complex}}~X~P_X~Y~P_Y~(B~x~y_1~y_2)
\end{infrule}
\begin{center}
\begin{infrule}
\forall\exists_{\text{Complex}}~X~P_X~Y~P_Y~c
====
\forall\exists_{\text{Complex}}~X~P_X~Y~P_Y~(C~y~c)
\end{infrule}
\end{center}


$\exists\forall$ can not be expressed as a simple inductive type due to the 
difficulty to transport the information which of the predicates $P_X$ or $P_Y$ 
is the one satisfied by all elements through recursion.
Therefore, $\exists\forall$ is expressed in terms of the other statements.
One of the predicates has to be fulfilled by all elements and the other
predicate can be ignore. Using this intuition, $\exists\forall$ is 
a disjunction of two applications of $\forall\forall$ with true predicates 
in place of the predicate not under observation.

\begin{infrule}
\forall\forall_{\text{Complex}}~X~P_X~Y~(\lambda \_.~\top)
====
\exists\forall_{\text{Complex}}~X~P_X~Y~P_Y~c
\end{infrule}
\begin{infrule}
\forall\forall_{\text{Complex}}~X~(\lambda \_.~\top)~Y~P_Y
====
\exists\forall_{\text{Complex}}~X~P_X~Y~P_Y~c
\end{infrule}

For each argument of one of the argument types as well as for
each a recursive instance a constructor is added with one proof
as argument.

\begin{infrule}
P_X~x
===
\exists\exists_{\text{Complex}}~X~P_X~Y~P_Y~(A~x)
\end{infrule}
\begin{infrule}
P_X~x
===
\exists\exists_{\text{Complex}}~X~P_X~Y~P_Y~(B~x~y_1~y_2)
\end{infrule}

\begin{infrule}
P_Y~y_1
===
\exists\exists_{\text{Complex}}~X~P_X~Y~P_Y~(B~x~y_1~y_2)
\end{infrule}
\begin{infrule}
P_Y~y_2
===
\exists\exists_{\text{Complex}}~X~P_X~Y~P_Y~(B~x~y_1~y_2)
\end{infrule}

\begin{infrule}
P_Y~y
===
\exists\exists_{\text{Complex}}~X~P_X~Y~P_Y~(C~y~c)
\end{infrule}
\begin{infrule}
\exists\exists_{\text{Complex}}~X~P_X~Y~P_Y~c
===
\exists\exists_{\text{Complex}}~X~P_X~Y~P_Y~(C~y~c)
\end{infrule}


\subsubsection{Guarded}

Guarded recursion is handled in the usual fashion.
But for the statements guarded instances of type arguments have to be considered.
\[ G~X~::=~C~(f:\mathbb{N}\to X)\]

\begin{center}
\begin{infrule}
\forall n.~P_X~(f~n)
====
\forall\forall_G~X~P_X~(C~f)
\end{infrule}
\end{center}

% TODO: 
\begin{center}
\begin{infrule}
\exists n.~P_X~(f~n)
===
\exists\exists_G~X~P_X~(C~f)
\end{infrule}
\end{center}


\subsubsection{Nested}

Nested in this context does not mean nested induction but rather nested container
occurrences in a container.

\[ R~X~::=~T~(xs:\text{List}~X) \]

In order to require all instances of type $X$ in an element of type $R$ to fulfill $P_X$,
all elements in the list $xs$ need to fulfill $P_X$.
The requirement that all elements in a list fulfill a predicate is expressed using $\forall\forall$.

\begin{center}
\begin{infrule}
\forall\forall_{\text{List}}~X~P_X~xs
====
\forall\forall_R~X~P_X~(T~xs)
\end{infrule}
\end{center}

In order to have at least one element in the instance of $R$ constructed by $T$ fulfilling 
$P_X$, an element in the list has to fulfill $P_X$.

\begin{center}
\begin{infrule}
\exists\exists_{\text{List}}~X~P_X~xs
====
\exists\exists_R~X~P_X~(T~xs)
\end{infrule}
\end{center}


\subsection{Comparison}

In the following table the structure is as follows:

\begin{center}
\fbox{\begin{minipage}{.5\textwidth}
\setlist[itemize]{wide=0pt, label=\textbullet, noitemsep, topsep=2pt, leftmargin=*}
    \begin{itemize}
\item statement: name in words
\item explanation of the meaning
\item translation structure
\item representation with other statements
    \end{itemize}
\end{minipage}}
\end{center}

\noindent
\newcolumntype{m}{>{\hsize=.5\hsize}X}
\begin{tabularx}{\textwidth}{ m|m }
$\forall\forall_T$: for all predicates for all elements = unary parametricity &
$\forall\exists_T$: for all predicates exists an element \\
All predicates are satisfied for all elements in the container &
All predicates are satisfied by at least one element in the container \\
One constructor is translated to one constructor &
\\
A conjunction of $\exists\forall_T$ for each predicate with $\bot$ in place of the other predicates &
A conjunction of $\exists\exists_T$ for each predicate with $\bot$ in place of the other predicates
\end{tabularx}

\noindent\makebox[\linewidth]{\rule{\textwidth}{0.4pt}}
\begin{tabularx}{\textwidth}{ m|m }
$\exists\forall_T$: exists predicates for all elements &
$\exists\exists_T$: exists predicates exists an element \\
At least one predicates is satisfied for all elements in the container &
At least one predicates is satisfied by at least one element in the container \\
&
one argument is translated to one constructor\\
A disjunction of $\forall\forall_T$ for each predicate with $\top$ in place of the other predicates &
A disjunction of $\forall\exists_T$ for each predicate with $\top$ in place of the other predicates
\end{tabularx}
\ \\
The conjunction can be represented by a single constructor with multiple arguments.
Similarly the disjunction can be represented by multiple single argument constructors.


\subsection{Construction}
The $\forall\forall$ and $\exists\exists$ have a common structure describing
their generation.

For $\forall\exists$ a generation procedure is possible but an inspection of 
all constructor arguments against the type arguments is needed which leads
to unintuitive results and complicated generation of the statements.

The $\exists\forall$ statement is difficult to express due to the difficulty
to choose one predicate and transport the choice through recursion.

Therefore, $\forall\exists$ and $\exists\forall$ should be expressed in terms
of $\forall\forall$ and $\exists\exists$ if needed.

 %cleaned
As $\forall\forall$ coincides with the trimmed unary parametricity translation 
of inductive types, the generation procedure is well known.

For $\exists\exists$ the general fulfillment criterion has to be considered:
If one argument of an argument type satisfies a predicate, the $\exists\exists$ 
statement is fulfilled. The other possibility to fulfill the $\exists\exists$ statement
is by recursion following a recursive argument.

From this intuition, the generation procedure for a container $T$ can be constructed:
\begin{itemize}
    \item for each constructor iterate over all arguments
    \item if the argument $x$ has one of the parametric types $X$ as type,
        add a constructor with an additional argument stating that
            $x$ fulfills the predicate $P_X$
    \item if the argument $t$ is recursive (of type $T$), add a constructor 
        with an additional argument stating that $\exists\exists$ is fulfilled for $t$
\end{itemize}

In contrast to $\forall\forall$ where one constructor is translated to 
one constructor with an additional argument for each suitable argument,
the translation for $\exists\exists$ converts each suitable argument into a new constructor.



%Todo: Describe Generation procedure
