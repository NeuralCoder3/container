%%%%%%%%%%%%%%%%%%%%%%%%%%%%%%%%%%%%%%%%%
% Journal Article
% LaTeX Template
% Version 1.4 (15/5/16)
%
% This template has been downloaded from:
% http://www.LaTeXTemplates.com
%
% Original author:
% Frits Wenneker (http://www.howtotex.com) with extensive modifications by
% Vel (vel@LaTeXTemplates.com)
%
% License:
% CC BY-NC-SA 3.0 (http://creativecommons.org/licenses/by-nc-sa/3.0/)
%
%%%%%%%%%%%%%%%%%%%%%%%%%%%%%%%%%%%%%%%%%

%----------------------------------------------------------------------------------------
%	PACKAGES AND OTHER DOCUMENT CONFIGURATIONS
%----------------------------------------------------------------------------------------

\documentclass{article}

\usepackage{blindtext} % Package to generate dummy text throughout this template 

\usepackage[sc]{mathpazo} % Use the Palatino font
\usepackage[T1]{fontenc} % Use 8-bit encoding that has 256 glyphs
\linespread{1.05} % Line spacing - Palatino needs more space between lines
\usepackage{microtype} % Slightly tweak font spacing for aesthetics

\usepackage[english]{babel} % Language hyphenation and typographical rules

\usepackage[hmarginratio=1:1,top=32mm,columnsep=20pt]{geometry} % Document margins
\usepackage[hang, small,labelfont=bf,up,textfont=it,up]{caption} % Custom captions under/above floats in tables or figures
\usepackage{booktabs} % Horizontal rules in tables

\usepackage{lettrine} % The lettrine is the first enlarged letter at the beginning of the text

\usepackage{enumitem} % Customized lists
\setlist[itemize]{noitemsep} % Make itemize lists more compact

\usepackage{abstract} % Allows abstract customization
\renewcommand{\abstractnamefont}{\normalfont\bfseries} % Set the "Abstract" text to bold
\renewcommand{\abstracttextfont}{\normalfont\small\itshape} % Set the abstract itself to small italic text

\usepackage{titlesec} % Allows customization of titles
\renewcommand\thesection{\arabic{section}} % Roman numerals for the sections
\renewcommand\thesubsection{\arabic{section}.\arabic{subsection}} % roman numerals for subsections
\titleformat{\section}[block]{\large\scshape\centering}{\thesection.}{1em}{} % Change the look of the section titles
\titleformat{\subsection}[block]{\large}{\thesubsection.}{1em}{} % Change the look of the section titles

\usepackage{fancyhdr} % Headers and footers
\pagestyle{fancy} % All pages have headers and footers
\fancyhead{} % Blank out the default header
\fancyfoot{} % Blank out the default footer
\fancyhead[C]{Running title $\bullet$ May 2016 $\bullet$ Vol. XXI, No. 1} % Custom header text
\fancyfoot[RO,LE]{\thepage} % Custom footer text

\usepackage{titling} % Customizing the title section

\usepackage{hyperref} % For hyperlinks in the PDF
\usepackage{amsmath}
\usepackage{tabularx}

%----------------------------------------------------------------------------------------
%	TITLE SECTION
%----------------------------------------------------------------------------------------

\setlength{\droptitle}{-4\baselineskip} % Move the title up

\pretitle{\begin{center}\Huge\bfseries} % Article title formatting
\posttitle{\end{center}} % Article title closing formatting
\title{Article Title} % Article title
\author{%
\textsc{Marcel Ullrich} \\[1ex] % Your name
\normalsize Saarland University \\ % Your institution
\normalsize \href{mailto:s8maullr@stud.uni-saarland.de}{s8maullr@stud.uni-saarland.de} % Your email address
%\and % Uncomment if 2 authors are required, duplicate these 4 lines if more
%\textsc{Jane Smith}\thanks{Corresponding author} \\[1ex] % Second author's name
%\normalsize University of Utah \\ % Second author's institution
%\normalsize \href{mailto:jane@smith.com}{jane@smith.com} % Second author's email address
}
\date{\today} % Leave empty to omit a date
\renewcommand{\maketitlehookd}{%
\begin{abstract}
\noindent \blindtext % Dummy abstract text - replace \blindtext with your abstract text
\end{abstract}
}

%----------------------------------------------------------------------------------------

\begin{document}

% Print the title
\maketitle


\section{Introduction}

\lettrine[nindent=0em,lines=3]{L} orem ipsum dolor sit amet, consectetur adipiscing elit.
\blindtext % Dummy text

\blindtext % Dummy text
\section{Statements}
The generalization of statements over container type uses different quantification
of the predicates and elements in its meaning.

Where unary parametricity is the forall predicate forall elements statement
which states that every predicate of the type arguments is satisfied for 
all elements in the container, other predicates using a notation
of exists are imaginable.

Especially the exists predicate exists element statement will be useful.
This statement postulates that for one of the predicate there is 
an element in the container satisfying the predicate.
For example $\exists\exists_{\text{List}}~X~P_X~xs$ states for a list $xs:\text{List}~X$
that there is an element $x\in xs$ with $P_X~x$.

Additionally, other combinations such as a $\exists\forall$ or $\forall\exists$ statement
are possible.

\subsection{Example statements}

We present the statement for common container type together with
their intuitive meaning.

\subsubsection{Product}
% X*X
% X*Y
We first look at a simpler type of products with only one type
argument.

\begin{infrule}
===
\forall\forall 
\end{infrule}

\subsubsection{List}

\subsubsection{Complex typ}

\subsubsection{Guarded}

\subsubsection{Nested}


\subsection{Comparison}

\newcolumntype{m}{>{\hsize=.5\hsize}X}
\begin{tabularx}{\textwidth}{ m|m }
$\forall\forall_T$: for all predicates for all elements = unary parametricity &
$\forall\exists_T$: for all predicates exists an element \\
All predicates are satisfied for all elements in the container &
All predicates are satisfied by at least one element in the container \\
One constructor is translated to one constructor &
\\
A conjunction of $\exists\forall_T$ for each predicate with $\bot$ in place of the other predicates &
A conjunction of $\exists\exists_T$ for each predicate with $\bot$ in place of the other predicates
\end{tabularx}

\noindent\makebox[\linewidth]{\rule{\textwidth}{0.4pt}}
\begin{tabularx}{\textwidth}{ m|m }
$\exists\forall_T$: exists predicates for all elements &
$\exists\exists_T$: exists predicates exists an element \\
At least one predicates is satisfied for all elements in the container &
At least one predicates is satisfied by at least one element in the container \\
&
one argument is translated to one constructor\\
A disjunction of $\forall\forall_T$ for each predicate with $\top$ in place of the other predicates &
A disjunction of $\forall\exists_T$ for each predicate with $\top$ in place of the other predicates
\end{tabularx}
\ \\
The conjunction can be represented by a single constructor with multiple arguments.
Similarly the disjunction can be represented by multiple single argument constructors.
\section{Subterm relation}
\section{Nested subterm}

To generate the nested subterm relation the $\exists\exists_T$ translation of the container
type $T$ is used.
\begin{thebibliography}{99} % Bibliography - this is intentionally simple in this template

% \bibitem[Figueredo and Wolf, 2009]{Figueredo:2009dg}
% Figueredo, A.~J. and Wolf, P. S.~A. (2009).
% \newblock Assortative pairing and life history strategy - a cross-cultural
%   study.
% \newblock {\em Human Nature}, 20:317--330.
 
\end{thebibliography}

\end{document}
