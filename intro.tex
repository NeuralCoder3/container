
\section{Introduction}

In type theory, types can have types as parametric arguments.
Element of those parametric types are built using elements
of the argument type. They encapsulate elements in them.
A prominent example are products and lists.
Products are constructed using an element of two 
arbitrary types each.
Lists on the other hand can contain arbitrary many
elements of one given type.

One often needs to argue about the elements inside
such a container type.
For example for nested induction a statement is 
needed that every element in the container fulfills
the induction hypothesis.
Another example are subterm relations of nested 
inductive types where an element in the argument list
of rose trees is a subterm of the whole rose tree.

It was discovered that the parametricity translation
provides the statement needed for nested induction.
The unary parametricity translation of a type is a predicate
on this type and all of its parametric type arguments
stating that certain properties are satisfied.
Therefore, the unary parametricity translation
can be seen as a statement that for all type arguments
and given predicates associated to this type
all elements in the container satisfy the predicates.
In short notation we will call this statement the
forall predicate forall element statement of a container type.

We will generalize statements over container types
and show applications related to the subterm relations.




