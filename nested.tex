\section{Nested subterm}

The subterm is generates as described in section~\ref{sec:subterm}.
The generation is extended to handle nested recursion.
Therefore, arguments of container types containing recursive instances
are captured by the nested subterm relation.

For arguments of container type $C$ the $\exists\exists_C$ statement is used 
to express the subterm relation.

\[\Rose~::=~\tree~(xs:\List~\Rose)\]

For rose trees each element in the argument list $xs$ is a subtree of the 
constructed tree $\tree xs$. To express the element that is considered to 
be a subterm, a new argument $r$ is introduced.
For the direct subterm relation, the subtree $x$ has to be one of the elements
in the list. This relation is expressed using equality.

\begin{einfrule}[c]
\exists\exists_{\List}~\Rose~(\lambda y.~x=y)~xs
=====
\Sub_{\Rose}~r~(\tree~xs)
\end{einfrule}

Alternatively, a transitive approach to nested subtrees is also
possible. To express transitive nested subterms, the subterm relation
is used as predicate which one element in the container has to satisfy.

\begin{einfrule}[c]
\exists\exists_{\List}~\Rose~(\Sub_{\Rose}~x)~xs
=====
\Sub_{\Rose}~r~(\tree~xs)
\end{einfrule}

\subsection{Examples}

The following examples highlight different possibilities how container 
could be instantiated and composed inside nested inductive types.

\subsubsection{Multiple recursive arguments}

\[T~::=~C~(p:\Prod~T~T)\]

An instance $t$ is a subterm of $C~p$ if it is one of the two elements
out of which the pair $p=(t_1,t_2)$ is composed.

\begin{infrule}
===
\Sub_T~t_1~(C~(t_1,t_2))
\end{infrule}
\begin{infrule}
===
\Sub_T~t_2~(C~(t_1,t_2))
\end{infrule}

But this approach only works for non-recursive container and need an
enumeration of all constructors. Therefore, the $\exists\exists$ 
statement is used to generalize the subterm generation.
An instance $t$ is a subterm of $C~p$ if one of the two types
have an element in the container which is equal to $t$, or simpler
$t$ is contained in $p$.

\begin{einfrule}[c]
\exists\exists_{\Prod}~T~(\lambda y.~t=y)~T~(\lambda y.~t=y)
===
\Sub_T~t~(C~p)
\end{einfrule}

\subsubsection{Non recursive arguments}

\[T~::=~C~(p:\Prod~\Nat~T)~|~D~(q:\Prod~\Nat~\Nat)\]

For the $D$ constructor, no subterm constructor is generated as 
$D$ has no recursive arguments.
The $C$ constructor has a pair with a subterm inside it.
To guarantee that $t$ is in the pair and therefore a subterm of $C~p$,
the other predicate, for the non recursive type $\Nat$, must not be satisfiable.
Therefore, the trivial false predicate $\lambda \_.~\bot$ is used.

\begin{center}
\begin{infrule}
\exists\exists_{\Prod}~\Nat~(\lambda \_.~\bot)~T~(\lambda y.~t=y)
===
\Sub_T~t~(C~p)
\end{infrule}
\end{center}

\subsubsection{Deep nesting}

\[T~::=~C~(p:\Prod~(\List~T)~T)\]

For deep nested container like the list of elements $t:T$ inside the pair $p$,
the $\exists\exists$ statements need to be nested.
A term $t$ is a subterm of $C~p$ is it is in the second component or an element
in the list which is the first component.
The element-of-list relation is expressed using $\exists\exists_{\List}$.

% eta conversion for inner \lambda xs. .... xs
\begin{center}
\begin{infrule}
\exists\exists_{\Prod}~(\List~T)~(\exists\exists_{\List}~T~(\lambda y.~t=y))~T~(\lambda y.~t=y)
===
\Sub_T~t~(C~p)
\end{infrule}
\end{center}


\subsection{Generation}

% recursive calls
In general 